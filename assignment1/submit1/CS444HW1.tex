\documentclass[letterpaper,10pt,fleqn]{article}

\usepackage{graphicx}
                                        
\usepackage{amssymb}                                         
\usepackage{amsmath}                                         
\usepackage{amsthm}                                          

\usepackage{alltt}                                           
\usepackage{float}
\usepackage{color}

\usepackage{url}

\usepackage{balance}
%\usepackage[TABBOTCAP, tight]{subfigure}
\usepackage{enumitem}

%\usepackage{pstricks, pst-node}

\usepackage{geometry}
\geometry{textheight=8.5in, textwidth=6in}

%random comment

\newcommand{\cred}[1]{{\color{red}#1}}
\newcommand{\cblue}[1]{{\color{blue}#1}}

\newcommand{\toc}{\tableofcontents}

%\usepackage{hyperref}

\def\name{Alec Zitzelberger and Chase Coltman}

%pull in the necessary preamble matter for pygments output
%\input{pygments.tex}

%% The following metadata will show up in the PDF properties
% \hypersetup{
%   colorlinks = false,
%   urlcolor = black,
%   pdfauthor = {\name},
%   pdfkeywords = {cs311 ``operating systems'' files filesystem I/O},
%   pdftitle = {CS 311 Project 1: UNIX File I/O},
%   pdfsubject = {CS 311 Project 1},
%   pdfpagemode = UseNone
% }

\parindent = 0.0 in
\parskip = 0.2 in

\author{A.\,N.~Other}
\title{Some things I did}

\begin{document}

\begin{titlepage}
    \begin{center}
        \vspace*{1cm}
        
        \textbf{CS444 Assignment 1}
        
        \vspace{0.5cm}
        Kernel, Concurrency and Latex
        
        \vspace{1.5cm}
        
        \textbf{Alec Zitzelberger and Chase Coltman}
        
        \vfill
        
       
        
        \vspace{0.8cm}
        
        
        %\includegraphics[width=0.4\textwidth]{university}
        
        Date October 9, 2017
        
    \end{center}
\end{titlepage}

\section* {Assignment 1}

\textbf{iqemu Steps Taken:}
	\begin{enumerate}
	\item Open up two shells, call them Shell A and Shell B.
	\item Git clone git://git.yoctoproject.org/linux-yocto-3.19
	\item Change extension to 3.19.2
	\item Be inside the linux-yocto-3.19.2 directory
	\item Mv ../../../files/bzImage-qemux86.bin .
	\item Mv ../../../files/config-3.19.2-yocto-standard .
	\item Mv ../../../files/environment-setup-i586-poky-linux .
	\item Enter bash for both Shell A and B
	\item Shell A: source environment-setup-i586-poky-linux
	\item Shell A: qemu-system-i386 -gdb tcp::5514 -S -nographic -kernel bzImage-qemux86.bin -drive file=core-image-lsb-sdk-qemux86.ext4,if=virtio -enable-kvm -net none -usb -localtime --no-reboot --append "root=/dev/vda rw console=ttyS0 debug".
	\item Shell B: source environment-setup-i586-poky-linux
	\item Shell B: \$GDB
	\item Shell B: Target remote :5514, then enter continue and hit enter
	\item Shell A: enter root and hit enter. Enter reboot and hit enter when ready to quit
	\item Hey it worked! Now to make our own
	\item Mv config-3.19.2-yocto-standard .config
	\item Make -j4 all
	\item Enter in make menuconfig
	\item Go to general and change the kernel info to group 14 homework 1
	\item Find bzImage in the folder arch/x86/boot/bzImage
	\item Cp arch/x86/boot/bzImage /scratch/fall2017/14/linux-yocto-3.19.2
	\item Shell A: source environment-setup-i586-poky-linux
	\item Shell A: qemu-system-i386 -gdb tcp::5514 -S -nographic -kernel bzImage -drive file=core-image-lsb-sdk-qemux86.ext4,if=virtio -enable-kvm -net none -usb -localtime --no-reboot --append "root=/dev/vda rw console=ttyS0 debug".
	\item Shell B: source environment-setup-i586-poky-linux
	\item Shell B: \$GDB
	\item Shell B: Target remote :5514
	\item Shell B: Continue
	\item Shell A: Root
	\item Logged into VM and ready to go
	\item Uname -a prints out our group number 14, homework 1
\end{enumerate}

\textbf{iqemu flags}
\begin{enumerate}

\item -gdb tcp::5514
	\begin{enumerate}
		\item 	This opens a gdbserver on TCP port 5514.
	\end{enumerate}

\item -S
	\begin{enumerate}
		\item 	This means Do not start CPU at startup.
	\end{enumerate}

\item -nographic
	\begin{enumerate}
		\item 	Disables graphical output so that Qemu is a simple command line application.
	\end{enumerate}

\item -kernel bzImage
	\begin{enumerate}
		\item 	This command says to use the bzImage as the kernel image, which can either be a linux kernel or in multiboot format.
	\end{enumerate}

\item -drive
	\begin{enumerate}
		\item file=core-image-lsb-sdk-qemux86.ext
			\begin{enumerate}
				\item This option defines which disk image to use with this drive. For this situation, the disk image would be core-image-lsb-sdk-qemux86.ext
			\end{enumerate}
		\item if=virtio
			\begin{enumerate}
				\item This option defines on which type of interface is connected. 
			\end{enumerate}
	\end{enumerate}

\item -enable-kvm
	\begin{enumerate}
		\item Enable KVM full virtualization support. This option is only available if KVM support is enabled when compiling
	\end{enumerate}

\item -net-none
	\begin{enumerate}
		\item Indicate that no network devices should be configured. It is used to override the default configuration which is activated if no -net options are provided
	\end{enumerate}

\item -usb
	\begin{enumerate}
		\item Simply enables the USB driver
	\end{enumerate}

\item -locatime
	\begin{enumerate}
		\item This is used to let the RTC start at the current local time
	\end{enumerate}

\item --no-reboot
	\begin{enumerate}
	\item Exit instead of rebooting
	\end{enumerate}

\item --append “root=/dev/vda rw console=ttyS0 debug”
	\begin{enumerate}
	\item This uses “root=/dev/vda rw console=ttyS0 debug” as the kernel command line
	\end{enumerate}

\end{enumerate}

\textbf{Concurrency Questions:}
\begin{enumerate}
	\item The point of this assignment has been to implement a multithreaded solution in order to better understand parallel programming. In addition, it has helped to familiarize the class with using the rdrand functionality on intel CPUs to produce “true” random numbers.
	\item I approached the problem by dividing up the functionality of the two thread types into two different functions. I implemented new code in waves. First was the threading, then mutex locking, semaphore blocking, etc. At each step, I tested to make sure that the layers were correctly coded. The algorithm was simple. Each function that the threads worked contained an infinite while loop that checked for the lock availability and did work based on the values detected in the global variables. If the buffer was empty or full, the while loop contained code to do the appropriate blocking. When running the code, users enter the number of consumers and producers. This makes our solution a multiple consumer and producer solution.
	\item To ensure the correctness, both myself and my partner tested the code at each stage of development. To be able to see the correctness, the program is littered with print statements that report what is going on for each thread. These print statements report the currently handled item number, the time spent “working” on it, and the current buffer contents.
	\item Most of what we learned involved just relearning things from OS1 like how to multi-thread, mutex lock, and using semaphores. The new content was primarily how to use the Twister number generator and the rdrand assembly code. The assembly code proved to be somewhat of a challenge when implementing in C, but it was good to learn.
\end{enumerate}

\textbf{Work.log for Chase Coltman}
\begin{enumerate}	
	\item Began working on installing Kernel and attempting to get it run. Worked through instructions until I got to
		\begin{enumerate}
			\item Steps Taken:
				\begin{enumerate}
					\item Created the repo and gave permissions for Alec to be able to enter and make changes
					\item Attempted to move files to the correct location
				\end{enumerate}
			\item Date - 9/30/2017
			\item Time - About 2 hours
			\item End-point - Frustrated and not sure what I am doing wrong
			\item Finished - No, the Kernel had a bunch of error so I waited until next class to ask a peer
		\end {enumerate}
		
	\item Attempted to work some more on getting Kernel to work
		\begin{enumerate}
			\item Steps Taken:
				\begin{enumerate}
					\item Moved all the files into the correct directory
					\item Seems to be working but not sure as it freezes when I run the code
				\end{enumerate}
			\item Date - 10/3/2017
			\item Time - About 2 hours
			\item End-point - Compiles and runs but freezes at the end
			\item Finished -  No, Kernel is compiling but not sure on next step
		\end {enumerate}
		
	\item Completed all the requirements for the Kernel to run
		\begin{enumerate}
			\item Steps Taken:
				\begin{enumerate}
					\item Finally got it working. Everything was actually working on the 3rd when I attempted to do it before, just had to open another shell client.
					\item Finished the implementation and did the write-up with all the steps that I took to figure it out
				\end{enumerate}
			\item Date - 10/5/2017
			\item Time - About 3-4 hours
			\item End-point - Runs and I am able to enter the kernel and VM with both the bzImage file and the one we make when we run make -j4 all
			\item Finished - Yes, It worked and I followed all the steps to make sure
		\end {enumerate}		
			
	\item Finished writing up what all the flags do
		\begin{enumerate}
			\item Steps Taken:
				\begin{enumerate}
					\item Googled qemu i386 flags to give me a detailed readme
					\item Found this link:http://www.tin.org/bin/man.cgisection=1\&topic=qemu-system-i386
					\item Wrote in what each flag did and learned what exactly we do when we run that long command
				\end{enumerate}
			\item Date - 10/7/2017
			\item Time - About 1 hour
			\item End-point - Finished
			\item Finished -  Yes, Reason for it being approximately a time is because I was at work and actually have no idea how long it took
		\end {enumerate}		
			
	\item Learning how to LaTeX
		\begin{enumerate}
			\item Steps Taken:
				\begin{enumerate}
					\item Watched the online tutorial
					\item Copy-pasted write up from Google Doc and formatted for LaTeX
				\end{enumerate}
			\item Date - 10/9/2017
			\item Time - About 2 hours
			\item End-point - Finished on my end, sending to Alec for his log and final product
			\item Finished -  No
		\end {enumerate}	
        \end {enumerate}
		
\textbf{Work.log for Alec Zitzelberger}
\begin{enumerate}	
	\item Began working on the Concurrency Solution
		\begin{enumerate}
			\item Steps Taken:
				\begin{enumerate}
					\item Researched how to use pthreading and mutexes in C
					\item Used the hellopthread.c code as a baseline for development as it contained a lot of syntax info
				\end{enumerate}
			\item Date - 10/5/2017
			\item Time - About 2 hours
			\item End-point - Had a decent understanding of the code libraries I would need for pthreading
			\item Finished - No, This would take more time to create the first alpha
		\end {enumerate}
		
	\item Finished working on the first version of the Concurrency Solution code
		\begin{enumerate}
			\item Steps Taken:
				\begin{enumerate}
					\item Continued to implement what I had learned during my research
					\item Finished writing the code for simple pthreading behavior and tested it locally
					\item Pushed the V1 to the MAIN repo
				\end{enumerate}
			\item Date - 10/6/2017
			\item Time - About 1 hours
			\item End-point - Had a basic version that chaotically did work in multiple threads
			\item Finished - Yes, completed the work phase I had begun the day before
		\end {enumerate}
		
	\item Worked on the second version of the Concurrency Solution code
		\begin{enumerate}
			\item Steps Taken:
				\begin{enumerate}
					\item Implemented mutex locking
					\item Tested the code that now locked the threads correctly
					\item Pushed V2 to the MAIN repo
				\end{enumerate}
			\item Date - 10/6/2017
			\item Time - About 1 hours
			\item End-point - Had code that followed the locking criteria of the solution
			\item Finished - Yes, completed the phase of coding mutexes
		\end {enumerate}
		
	\item Worked on the third version of the Concurrency Solution code
		\begin{enumerate}
			\item Steps Taken:
				\begin{enumerate}
					\item Researched semaphore signalling for C code
					\item Implemented semaphore blocking
					\item Tested the code that now blocked the threads correctly given the state of the buffer
					\item Pushed V3 to the MAIN repo
				\end{enumerate}
			\item Date - 10/6/2017
			\item Time - About 1 hours
			\item End-point - Had code that followed the blocking criteria of the solution
			\item Finished - Yes, completed the phase of coding conditional thread blocking
		\end {enumerate}
		
	\item Worked on the fourth version of the Concurrency Solution code
		\begin{enumerate}
			\item Steps Taken:
				\begin{enumerate}
					\item Researched rdrand and Mersenne Twister random number generation for C code
					\item Implemented random number generation
					\item Tested the code that now generated random numbers locally
					\item Pushed V4 to the MAIN repo
				\end{enumerate}
			\item Date - 10/8/2017
			\item Time - About 2 hours
			\item End-point - Had code that followed the random number criteria of the solution
			\item Finished - Yes, completed the phase of coding random number generation
		\end {enumerate}
		
	\item Worked on the fifth version of the Concurrency Solution code
		\begin{enumerate}
			\item Steps Taken:
				\begin{enumerate}
					\item Researched pthread wait blocking for C code
					\item Implemented pthread wait blocking
					\item Tested the code that now blocked the threads correctly given the state of the buffer
					\item Pushed V5 to the MAIN repo
				\end{enumerate}
			\item Date - 10/9/2017
			\item Time - About 1 hours
			\item End-point - Had code that followed the blocking criteria of the solution
			\item Finished - Yes, completed the phase of coding conditional thread blocking
		\end {enumerate}
        \end{enumerate}
		
		
\begin{table}[h]
  \begin{tabular}{|l|l|l|}
  \hline
  Date & Action & Version \\ \hline
 9:00 10/5/2017 & Created Project with basic threading in local repository. & V1 \\ 
 14:33 10/6/2017 & Pushed Project to MAIN repository. & V1 \\ 
 15:11 10/6/2017 & Added Mutex locking local repository. & V2 \\ 
 15:16 10/6/2017 & Pushed Project to MAIN repository. & V2 \\ 
 16:35 10/6/2017 & Added semaphore blocking in local repository. & V3 \\ 
 16:42 10/6/2017 & Pushed Project to MAIN repository. & V3 \\ 
 14:03 10/8/2017 & Added rdrand and Mersenne in local repository. & V4 \\ 
 17:24 10/8/2017 & Pushed Project to MAIN repository. & V4 \\ 
 10:02 10/9/2017 & Replaced semaphore with pthread\_wait\_conditionals in local repository. & V5 \\ 
 10:10 10/9/2017 & Pushed Project to MAIN repository. & V5 \\
  \hline
  \end{tabular}
  \end{table}

\end{document}
