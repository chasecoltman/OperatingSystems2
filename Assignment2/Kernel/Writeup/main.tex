\documentclass[journal,10pt,onecolumn,compsoc]{IEEEtran} \usepackage[margin=1.0in]{geometry} \usepackage{pdfpages} 
\usepackage{caption,graphicx,float} 
\usepackage{listings}
\usepackage{verbatim}
\usepackage{enumitem}
\usepackage{url}
\usepackage{amsmath}
\graphicspath{/graphics} \setlength{\parskip}{\baselineskip} \setlength\parindent{24pt}
\usepackage[english]{babel}
%\usepackage{fullpage}

\title{TensorFlow\texttrademark WYSIWYG GUI Design Document}
\author{Group 14: Chase Coltman, Alec Zitzelberger}
\date{\today}
\begin{document}

\begin{titlepage}
    \begin{center}
        \vspace*{1cm}
        
        \textbf{CS444 Program Assignment 2}
        
        \vspace{0.5cm}
        I/O Elevators
        
        \vspace{1.5cm}
        
        \textbf{Alec Zitzelberger and Chase Coltman}
        
        \vfill
        
       
        
        \vspace{0.8cm}
        
        
       % \includegraphics[width=0.4\textwidth]{university}
        
        Date October 30, 2017
        
    \end{center}
\end{titlepage}

\section {Questions}
\begin {enumerate}
\item What do you think the main point of this assignment is?
	\begin{itemize}
    	\item We believe that the main point of this assignment was to get a better understanding of how our computers handle CPU scheduling, as well as learn how to implement multiple different versions of them.
    \end{itemize}
\item How did you personally approach the problem? Design decisions, algorithm, etc.
	\begin{itemize}
    	\item We approached this problem by first studying the difference between noop and look. Once we did this, we went in and converted as much of the noop to the look algorithm as we could as several of the functions are very similar with only minimal changes like naming conventions and changing the names of the ops. After that, we attempted to get the kernel to recognize our particular scheduler, and once we did we attempted to find out how to test that.
    \end{itemize}
\item How did you ensure your solution was correct? Testing details, for instance.
	\begin{itemize}
    	\item The way that we tested our program was to verify that the correct scheduler was getting selected when we ran our qemu command. Inside the shell, we used cat /sys/block/hdc/queue/scheduler. When we do this, we are supposed to get a message telling us which schedulers there are, and the one selected is displayed in brackets. We noticed that not only was look one of the possible schedulers, but it was the one with brackets around it. This told us that our scheduler was compiling correctly and was being selected as the IO scheduler. 
    \end{itemize}
\item What did you learn?
	\begin{itemize}
    	\item We learned that finding information on Kernel manipulation is surprisingly hard to do. Most people online with experience in it simply say don't do it, which is challenging for trying to learn. However, there was a lot of resources that explain the process that we were able to translate into code. Creating an I/O scheduler proved to be a bit harder then we had anticipated.  
    \end{itemize}
\item How should the TA evaluate your work? Provide detailed steps to prove correctness.
	\begin{itemize}
    	\item TEMP LOREM IPSUM
    \end{itemize}

\end{enumerate}

\section {Work.log for Alec Zitzelberger}

\section{Work.log for Chase Coltman}

\section{Commit.log GitHub}
 
\end{document}`'
